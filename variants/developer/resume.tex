\documentclass[]{mcdowellcv}

\usepackage{graphicx}
% For mathematical symbols
\usepackage{amsmath}
\usepackage[hidelinks,draft=false]{hyperref}
\RequirePackage{xcolor}
\RequirePackage{fontawesome}
\colorlet{accent}{blue!70!black}

\hypersetup{final}

\newcommand{\printinfo}[2]{\mbox{\textcolor{accent}{\normalfont #1}\hspace{0.5em}#2\hspace{2em}}}
\newcommand{\emailsymbol}{\faAt}
\newcommand{\phonesymbol}{\faPhone}
\newcommand{\mailsymbol}{\faEnvelope}
\newcommand{\linkedinsymbol}{\faLinkedin}
\newcommand{\githubsymbol}{\faGithub}
\newcommand{\email}[1]{\printinfo{\emailsymbol}{#1}}
\newcommand{\phone}[1]{\printinfo{\phonesymbol}{#1}}
\newcommand{\mailaddress}[1]{\printinfo{\mailsymbol}{#1}}
\newcommand{\github}[1]{\printinfo{\githubsymbol}{#1}}
\newcommand{\linkedin}[1]{\printinfo{\linkedinsymbol}{#1}}
\newcommand\chref[3][accent]{\href{#2}{\small\color{#1}#3}}


%! Author = abhi
%! Date = 12/31/25

% ================================
% Common header (shared by all resumes)
% ================================

\name{Abhishek Kumar}

\contacts{\email{\href{mailto:abhi@apache.org}{abhi@apache.org}}}

\gitlink{\github{\href{https://github.com/kumaab}{github.com/kumaab}}}

\address{\phone{+1-434-995-9140}}

\webpage{\linkedin{\href{https://www.linkedin.com/in/kumar21}{linkedin.com/in/kumar21}}}

\begin{document}

    % Print the header
    \makeheader
    % Print the content
    \begin{cvsection}{Work Experience}
        \begin{cvsubsection}{Senior Software Engineer}{Cloudera, Santa Clara}{Feb 2024 - Present}
            \begin{itemize}
                \item Co-led cross-team initiatives to implement cloud-native fine-grained authorization (Ranger RAZ) for Cloudera AI Inference service.
                \item Developed a distributed performance testing framework using Apache JMeter to identify performance bottlenecks in custom internal authorizers.
                \item Co-led development efforts to revamp Apache Ranger codebase to conform to check-style changes \chref{https://issues.apache.org/jira/browse/RANGER-5017}{RANGER-5017}.
            \end{itemize}
        \end{cvsubsection}
        \begin{cvsubsection}{Apache Ranger PMC Member & Committer}{Apache Software Foundation}{Feb 2023 - Present}
            \begin{itemize}
                \item Led end-to-end release activities for Apache Ranger 2.6.0 and 2.7.0 as release manager, coordinating 200+ contributions while ensuring production readiness.
                \item Presented a talk on \'Unifying Access Control with Governed Data Sharing in Apache Ranger\' at Apache's official conference Community Over Code North America, held in Minneapolis, Minnesota in Sept' 2025.
            \end{itemize}
        \end{cvsubsection}
        \begin{cvsubsection}{Software Engineer II}{Cloudera, Santa Clara}{Feb 2021 - Jan 2024}
            \begin{itemize}
                \item \chref{https://github.com/apache/ranger}{\textbf{Apache Ranger}} Committer : Landed 100+ commits in areas including docker development, feature development for ranger usersync service and various bug fixes across the product.
                \item Lead developer for the ranger usersync service, driving feature development and actively addressing customer escalations(resolved 30+ relevant tickets in the last 2 years).
                \item Led docker-based development initiatives within the product in an effort to support all plugin services and reduce feature-related developer testing time.
            \end{itemize}
        \end{cvsubsection}

        \begin{cvsubsection}{Software Engineer}{Goldman Sachs, Bengaluru}{June 2017 - July 2019}
            \begin{itemize}
                \item Optimized performance of a critical component in a legacy distributed system \chref{https://www.google.com/patents/US20130275612}{Gigabus}, increasing its throughput by 36\%.
                \item Developed materialized views for read queries on Sybase IQ, the slowest performing views went from 25 mins to 5 mins.
            \end{itemize}
        \end{cvsubsection}
    \end{cvsection}

    \begin{cvsection}{Internship}
        \begin{cvsubsection}{Software Engineering Intern}{Cloudera, Santa Clara}{June 2020 – Aug 2020}
            \begin{itemize}
                \item Developed client libraries in Java and Python for RESTful web services in Apache Ranger. \chref{https://github.com/apache/ranger/tree/master/intg}{[Source Code]}
            \end{itemize}
        \end{cvsubsection}
        \begin{cvsubsection}{Summer Intern}{Goldman Sachs, Bengaluru}{May 2016 - July 2016}
            \begin{itemize}
                \item Built and deployed workbooks in Tableau to query Sybase IQ and gain insights.
            \end{itemize}
        \end{cvsubsection}
    \end{cvsection}

    \begin{cvsection}{Education}
        \begin{cvsubsection}{Stony Brook University}{Stony Brook, NY}{Aug 2019 -- Dec 2020}
            \begin{itemize}
                \item {M.S  in Computer Science \hspace{32em} GPA: 3.7/4.0}
                \item \textbf{Coursework:} Distributed Systems, Analysis of Algorithms, Artificial Intelligence, Machine Learning, Probability and Statistics for Data Scientists, Databases
            \end{itemize}
        \end{cvsubsection}
    \end{cvsection}

    \begin{cvsection}{Technical Skills}
        \begin{cvsubsection}{}{}{}
            \begin{itemize}
                \item \textbf{Languages: } \{Java, Python\} (proficient), \{C, C++, Go, SQL, Bash\} (familiar)
                \item \textbf{Technologies \& Frameworks: } Kubernetes, Docker, Git, GitHub, Spring, Maven, IntelliJ, Redis, Cadence, Apache Ozone, Apache Atlas, Apache Jmeter
            \end{itemize}
        \end{cvsubsection}
    \end{cvsection}

    \begin{cvsection}{Academic Projects}{}
        \begin{cvsubsection}{}{}{}
            \begin{itemize}
                \item  \textbf{Fault-tolerant Key/Value Service with Raft | \textit{Distributed Systems, Golang}}: Implemented the \chref{https://raft.github.io/raft.pdf}{RAFT} (a replicated state machine protocol) consensus algorithm in Golang from scratch.
                Built a fault-tolerant key-value service on top of RAFT. 2019
            \end{itemize}
        \end{cvsubsection}
    \end{cvsection}
\end{document}
