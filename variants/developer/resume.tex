\documentclass[]{mcdowellcv}

\usepackage{graphicx}
% For mathematical symbols
\usepackage{amsmath}
\usepackage[hidelinks,draft=false]{hyperref}
\RequirePackage{xcolor}
\RequirePackage{fontawesome}
\colorlet{accent}{blue!70!black}

\hypersetup{final}

\newcommand{\printinfo}[2]{\mbox{\textcolor{accent}{\normalfont #1}\hspace{0.5em}#2\hspace{2em}}}
\newcommand{\emailsymbol}{\faAt}
\newcommand{\phonesymbol}{\faPhone}
\newcommand{\mailsymbol}{\faEnvelope}
\newcommand{\linkedinsymbol}{\faLinkedin}
\newcommand{\githubsymbol}{\faGithub}
\newcommand{\email}[1]{\printinfo{\emailsymbol}{#1}}
\newcommand{\phone}[1]{\printinfo{\phonesymbol}{#1}}
\newcommand{\mailaddress}[1]{\printinfo{\mailsymbol}{#1}}
\newcommand{\github}[1]{\printinfo{\githubsymbol}{#1}}
\newcommand{\linkedin}[1]{\printinfo{\linkedinsymbol}{#1}}
\newcommand\chref[3][accent]{\href{#2}{\small\color{#1}#3}}


%! Author = abhi
%! Date = 12/31/25

% ================================
% Common header (shared by all resumes)
% ================================

\name{Abhishek Kumar}

\contacts{\email{\href{mailto:abhi@apache.org}{abhi@apache.org}}}

\gitlink{\github{\href{https://github.com/kumaab}{github.com/kumaab}}}

\address{\phone{+1-434-995-9140}}

\webpage{\linkedin{\href{https://www.linkedin.com/in/kumar21}{linkedin.com/in/kumar21}}}

\begin{document}

    % Print the header
    \makeheader
    % Print the content

    \begin{cvsection}{Education}
        \begin{cvsubsection}{Stony Brook University}{Stony Brook, NY}{Aug 2019 -- Dec 2020}
            \begin{itemize}
                \item {M.S  in Computer Science \hspace{32em} GPA: 3.70/4.0}
                \item \textbf{Coursework:} Distributed Systems, Analysis of Algorithms, Databases, Machine Learning, Probability and Statistics for Data Scientists, Artificial Intelligence
                \item \textbf{Graduate Teaching Assistant}: Statistical Methods for Data Science (Fall \'20), Principles of Database Systems (Spring \'20)
            \end{itemize}
        \end{cvsubsection}
        \begin{cvsubsection}{Birla Institute of Technology Mesra}{Ranchi, India}{Aug 2013 -- June 2017}
            \begin{itemize}
                \item {B.E in Information Technology  \hspace{30em}}
                \item \textbf{Coursework:} Operating Systems, Data Structures, Algorithms, Data Mining, Databases, Optimization Techniques
            \end{itemize}
        \end{cvsubsection}
    \end{cvsection}

    \begin{cvsection}{Technical Skills}
        \begin{cvsubsection}{}{}{}
            \begin{itemize}
                \item \textbf{Languages: } \{Java, Python\} (proficient), \{C, C++, Go, SQL, Bash\} (familiar)
                \item \textbf{Technologies \& Frameworks: } Kubernetes, Docker, Git, GitHub, Spring, Maven, IntelliJ, Redis, Cadence, Apache Jmeter
            \end{itemize}
        \end{cvsubsection}
    \end{cvsection}

    \begin{cvsection}{Work Experience}
        \begin{cvsubsection}{Senior Software Engineer}{Cloudera, Santa Clara}{Feb 2024 - Present}
            \begin{itemize}
                \item Led the release activities for Apache Ranger 2.6 release.
                \item Co-led development efforts to revamp Apache Ranger codebase to conform to check-style changes \chref{https://issues.apache.org/jira/browse/RANGER-5017}{RANGER-5017}.
                \item Developed a new internal distributed performance testing framework leveraging Apache Jmeter to measure performance bottlenecks in an external authorizer used within Apache Hadoop.
            \end{itemize}
        \end{cvsubsection}
        \begin{cvsubsection}{Software Engineer II}{Cloudera, Santa Clara}{Feb 2021 - Jan 2024}
            \begin{itemize}
                \chref{https://github.com/apache/ranger}{\textbf{Apache Ranger}} Committer : Landed 100+ commits in areas including docker development, feature development for ranger usersync service and various bug fixes across the product.
                \item Lead developer for the ranger usersync service, driving feature development and actively addressing customer escalations(resolved 30+ relevant tickets in the last 2 years).
                \item Led docker-based development initiatives within the product in an effort to support all plugin services and reduce feature-related developer testing time.
            \end{itemize}
        \end{cvsubsection}

        \begin{cvsubsection}{Software Engineer}{Goldman Sachs, Bengaluru}{June 2017 - July 2019}
            \begin{itemize}
                \item Optimized performance of a critical component in a legacy distributed system \chref{https://www.google.com/patents/US20130275612}{Gigabus}, increasing its throughput by 36\%.
                \item Developed materialized views for read queries on Sybase IQ, the slowest performing views went from 25 mins to 5 mins.
                \item Migrated internal proprietary code base onto v16 Sybase IQ clusters which resulted in a 15\% gain on read SQL queries.
            \end{itemize}
        \end{cvsubsection}
    \end{cvsection}

    \begin{cvsection}{Internship}
        \begin{cvsubsection}{Software Engineering Intern}{Cloudera, Santa Clara}{June 2020 – Aug 2020}
            %iChat AV
            \begin{itemize}
                \item Built client libraries for the open source project Apache Ranger in Java and Python for RESTful web services enabling applications to manage ranger policies.\chref{https://github.com/apache/ranger/tree/master/intg}{[Source Code]}
            \end{itemize}
        \end{cvsubsection}

        \begin{cvsubsection}{Summer Intern}{Goldman Sachs, Bengaluru}{May 2016 - July 2016}
            \begin{itemize}
                \item Built and deployed workbooks in Tableau to query Sybase IQ and gain insights into the data. Extended the functionality to web based data sources.
            \end{itemize}
        \end{cvsubsection}
    \end{cvsection}


    \begin{cvsection}{Academic Projects}{}
        \begin{cvsubsection}{}{}{}
            \begin{itemize}
                \item  \textbf{Fault-tolerant Key/Value Service with Raft | \textit{Distributed Systems, Go}}: Implemented the \chref{https://raft.github.io/raft.pdf}{raft} (a replicated state machine protocol) consensus algorithm in Golang from scratch. Built a fault-tolerant key-value service on top of raft. (Nov'19)
                \item \textbf{Inverted Index generation using MapReduce | \textit{Distributed Systems, Go}}: Implemented a distributed MapReduce in Golang (based on the original \chref{http://static.googleusercontent.com/media/research.google.com/en//archive/mapreduce-osdi04.pdf}{\textit{research paper}}) to generate inverted indices for document searching.
                \chref{https://github.com/kumaab/MapReduce}{[Source Code]} (Sept 2019)
            \end{itemize}
        \end{cvsubsection}
    \end{cvsection}

    \begin{cvsection}{Academic Experience and Activities}
        \begin{cvsubsection}{}{}{}
            \begin{itemize}
                \item \textbf{Graduate Aptitude Test in Engineering - Computer Science (GATE-2017)}
                \begin{itemize}
                    \item Achieved a percentile of 99.3, an examination conducted jointly by IISc and IITs across India.
                \end{itemize}
            \end{itemize}
        \end{cvsubsection}
    \end{cvsection}
\end{document}
